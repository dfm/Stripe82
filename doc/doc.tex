\documentclass[12pt,preprint]{aastex}

% has to be before amssymb it seems
\usepackage{color,hyperref}
\definecolor{linkcolor}{rgb}{0,0,0.5}
\hypersetup{colorlinks=true,linkcolor=linkcolor,citecolor=linkcolor,
            filecolor=linkcolor,urlcolor=linkcolor}
\newcommand{\numberparagraphs}{}
\newcommand{\nonumberparagraphs}{}

\usepackage{url}
\usepackage{algorithmic,algorithm}
\usepackage{amssymb,amsmath}

\newcommand{\arxiv}[1]{\href{http://arxiv.org/abs/#1}{arXiv:#1}}

\usepackage{listings}
\definecolor{lbcolor}{rgb}{0.9,0.9,0.9}
\lstset{language=Python,
        basicstyle=\footnotesize\ttfamily,
        showspaces=false,
        showstringspaces=false,
        tabsize=2,
        breaklines=false,
        breakatwhitespace=true,
        identifierstyle=\ttfamily,
        keywordstyle=\bfseries\color[rgb]{0.133,0.545,0.133},
        commentstyle=\color[rgb]{0.133,0.545,0.133},
        stringstyle=\color[rgb]{0.627,0.126,0.941},
    }

\newcommand{\project}[1]{{\sffamily #1}}
\newcommand{\Python}{\project{Python}}
\newcommand{\github}{\project{GitHub}}
\newcommand{\thisplain}{emcee}
\newcommand{\this}{\project{\thisplain}}
\newcommand{\paper}{article}

\newcommand{\foreign}[1]{\emph{#1}}
\newcommand{\etal}{\foreign{et\,al.}}
\newcommand{\etc}{\foreign{etc.}}

\newcommand{\Fig}[1]{Figure~\ref{fig:#1}}
\newcommand{\fig}[1]{\Fig{#1}}
\newcommand{\figlabel}[1]{\label{fig:#1}}
\newcommand{\Tab}[1]{Table~\ref{tab:#1}}
\newcommand{\tab}[1]{\Tab{#1}}
\newcommand{\tablabel}[1]{\label{tab:#1}}
\newcommand{\Eq}[1]{Equation~(\ref{eq:#1})}
\newcommand{\eq}[1]{\Eq{#1}}
\newcommand{\eqlabel}[1]{\label{eq:#1}}
\newcommand{\Sect}[1]{Section~\ref{sect:#1}}
\newcommand{\sect}[1]{\Sect{#1}}
\newcommand{\App}[1]{Appendix~\ref{sect:#1}}
\newcommand{\app}[1]{\App{#1}}
\newcommand{\sectlabel}[1]{\label{sect:#1}}
\newcommand{\Algo}[1]{Algorithm~\ref{algo:#1}}
\newcommand{\algo}[1]{\Algo{#1}}
\newcommand{\algolabel}[1]{\label{algo:#1}}

% math symbols
\newcommand{\dd}{\mathrm{d}}
\newcommand{\bvec}[1]{\ensuremath{\boldsymbol{#1}}}
\newcommand{\paramvector}[1]{\bvec{#1}}
\newcommand{\normal}[1]{\ensuremath{\mathcal{N} \left (#1\right )}}

\renewcommand{\vector}[1]{\bvec{#1}}
\renewcommand{\matrix}[1]{\ensuremath{#1}}

\newcommand{\pr}[1]{\ensuremath{p(#1)}}

% model parameters
\newcommand{\nstars}{\ensuremath{M}}
\newcommand{\nobs}  {\ensuremath{N}}
\newcommand{\st}    {\ensuremath{\alpha}}
\newcommand{\obs}   {\ensuremath{i}}
\newcommand{\jabs}  {\ensuremath{\delta_\obs}}
\newcommand{\vs}    {\ensuremath{\eta_\st}}
\newcommand{\vo}    {\ensuremath{\beta_\obs}}
\newcommand{\sigobs}{\ensuremath{\sigma_{\obs\st}}}
\newcommand{\sig}   {\ensuremath{\Sigma_{\obs\st}}}
\newcommand{\fstar} {\ensuremath{f_\st}}
\newcommand{\fobs}  {\ensuremath{z_\obs}}
\newcommand{\Cobs}  {\ensuremath{C_{\obs\st}}}
\newcommand{\Cmod}  {\ensuremath{\bar{C}_{\obs\st}}}

% units
\newcommand{\unit}[1]{\mathrm{#1}}

\begin{document}

\title{Time domain photometric calibration}

\newcommand{\nyu}{2}
\newcommand{\mpia}{3}
\author{Daniel~Foreman-Mackey\altaffilmark{1,\nyu},
    David~W.~Hogg\altaffilmark{\nyu,\mpia}, \etal}
\altaffiltext{1}{To whom correspondence should be addressed:
                        \url{danfm@nyu.edu}}
\altaffiltext{\nyu}{Center for Cosmology and Particle Physics,
                        Department of Physics, New York University,
                        4 Washington Place, New York, NY, 10003, USA}
\altaffiltext{\mpia}{Max-Planck-Institut f\"ur Astronomie,
                        K\"onigstuhl 17, D-69117 Heidelberg, Germany}

\begin{abstract}
\end{abstract}

\keywords{
}

\section{The model}

For one patch on the sky with \nobs\ runs (indexed by \obs ) and \nstars\
stars (indexed by \st ), we model the observed fluxes (\Cobs ) of the
stars as the product of the true flux of the star \fstar\ and the zero-point
offset of the run \fobs. We model the
observation as being being drawn from the distribution
\begin{equation}
    \Cobs \sim \normal{\Cmod, \sig^2}
\end{equation}
where
\begin{equation}
    \Cmod \equiv \fstar \, \fobs
\end{equation}
and $\sig^2$ is a physically motivated noise model with several components.
Since stars in the patch could be intrinsically variable, each star is
given a multiplicative variance $\vs^2$ and since any of the runs could be
noisy or ``bad'', each run has an associated variance $\jabs^2$ and a
multiplicative variance $\vo^2$. Therefore, the
noise model is
\begin{equation}
    \sig^2 = \sigobs^2 + \jabs^2 + \left ( \vo^2 + \vs^2 \right ) \, \Cmod^2
\end{equation}
where \sigobs\ is the quoted observational uncertainty. The negative
log-likelihood function (the ``objective function'' that we would like to
minimize) associated with this model is
\begin{equation}\eqlabel{nll}
    \ell (\theta) = \frac{1}{2} \sum_{\st=1}^\nstars \sum_{\obs=1}^\nobs
        \left [
            \frac{(\Cobs - \Cmod)^2}{\sig^2} + \ln \sig^2
        \right ]
\end{equation}
where $\theta = \{ \fstar, \fobs, \jabs^2, \vo^2, \vs^2 \}$ is the set of
$2\nstars + 3\nobs$ model parameters that we will be optimizing. The
optimal set of parameters
\begin{equation}
    \theta^* = \operatorname*{arg\,min}_\theta \ell (\theta)
\end{equation}
is equivalent to the \emph{maximum likelihood} setting of the parameters.
It is worth noting that the second term in \eq{nll} ``prefers'' solutions
with smaller variance. In other words, the optimal values for $\jabs^2$,
$\vo^2$ and $\vs^2$ will be as small as possible while still describing the
data well. This is similar to applying an L1 regularization to the objective
but in fact, it is a necessary component of the properly normalized
likelihood function itself.

It is also useful to calculate the gradient of $\ell (\theta)$ for the
purposes of optimization. First, the derivative with respect to $\sig^2$ is
\begin{equation}
    \frac{\dd \ell (\theta)}{\dd \sig^2} =
        \frac{1}{2\,\sig^2} - \frac{(\Cobs - \Cmod)^2}{2 \sig^4} \quad .
\end{equation}
From this, we find the other components of the gradient
\begin{equation}
    \frac{\dd \ell (\theta)}{\dd \jabs^2} = \sum_{\st=1}^\nstars
        \frac{\dd \ell (\theta)}{\dd \sig^2}
    \quad , \quad
    \frac{\dd \ell (\theta)}{\dd \vo^2} = \sum_{\st=1}^\nstars
        \Cmod^2 \, \frac{\dd \ell (\theta)}{\dd \sig^2}
    \quad \mathrm{and} \quad
    \frac{\dd \ell (\theta)}{\dd \vs^2} = \sum_{\obs=1}^\nobs
        \Cmod^2 \, \frac{\dd \ell (\theta)}{\dd \sig^2}
    \quad .
\end{equation}
Then,
\begin{equation}
    \frac{\dd \ell(\theta)}{\dd \Cmod} =
        2 \, \Cmod \, \left (\vo^2 + \vs^2\right ) \,
        \frac{\dd \ell (\theta)}{\dd \sig^2}
        - \frac{\Cobs - \Cmod}{\sig^2}
\end{equation}
giving
\begin{equation}
    \frac{\dd \ell (\theta)}{\dd \fstar} = \sum_{\obs=1}^\nobs
        \frac{\dd \ell(\theta)}{\dd \Cmod} \, \fobs
    \quad \mathrm{and} \quad
    \frac{\dd \ell (\theta)}{\dd \fobs} = \sum_{\st=1}^\nstars
        \frac{\dd \ell(\theta)}{\dd \Cmod} \, \fstar
    \quad .
\end{equation}
We optimize this using L-BFGS-B constrained optimization for non-negativity
on all the parameters.

% \begin{thebibliography}{}\raggedright

% \end{thebibliography}

\end{document}


