\documentclass[12pt]{article}

\newcommand{\project}[1]{\textsl{#1}}
\newcommand{\ps}{\project{PanSTARRS}}
\newcommand{\sdss}{\project{SDSS}}
\newcommand{\stripe}{\project{SDSS Stripe~82}}
\newcommand{\rrl}{RR~\textit{Lyrae}}

\begin{document}

\paragraph{Notes about finding \rrl\ in \project{PanSTARRS}}
~ \\
\textsl{by} DWH, HWR, DFM, others

\paragraph{introduction:}
The multi-band three-pi survey of \ps\ has great promise for finding
\rrl\ stars:
\begin{itemize}
\item \ps\ \project{3-pi} imaging is wider-field and deeper than
  \sdss\ imaging.  \sdss\ imaging already has produced extremely
  valuable \rrl\ samples; \ps\ ought to produce a much better sample,
  extending further and wider in the Galaxy.
\item The \ps\ \project{3-pi} Survey has multiple epochs taken in
  multiple filters over a few years.  This is better cadence than most
  of \sdss\ (though worse than \stripe).
\item The imaging in the different \ps\ filters is not
  taken simultaneously; this makes analysis of multi-band variability
  less trivial, but does not reduce the information content.  In
  particular, because \rrl\ stars have predictable color variability,
  this non-simultaneity is either no disadvantage or else a
  substantial advantage (when evaluated against other options at fixed
  telescope time).
\end{itemize}

\paragraph{photometric loops:}
\rrl\ stars are not perfect periodic oscillators; they show amplitude,
phase, and lightcurve-shape variations over time.  However, studies of
\stripe\ \rrl\ stars show that over durations of
months to years, they can be reasonably fit with periodic models (cite
S\'esar).  In the periodic approximation, in every photometric
bandpass, the star oscillates in a predictable way, returning every
cycle to the same brightness (in the absence of observational noise).

In a survey in which we measure the stars in $D$ photometric
bandpasses, the lightcurve can be thought of as a trajectory in a
$D$-dimensional space of the $D$ photometric measurements (think of
the measurements as components of a $D$-vector).  In the periodic
approximation, the trajectory is some kind of (possibly complicated)
closed one-dimensional loop in the $D$-space (in the absence of
observational noise).

\textbf{The first order of business might be to fit some kind of flexible
photometric-loop model to the photometry of various known \rrl\ stars
in \stripe\ data.}

There are two simple ways to go beyond the pure periodic
approximation.  The first, which is trivial but limited, is to model
the lightcurves with a periodic model but permit a substantial ``model
scatter''.  In this approach, each observation is treated as being
drawn from a the appropriate point on the photometric loop plus a
Gaussian noise that includes both observational noise \emph{and} model
scatter.  This is extremely easy to implement, but of course it
doesn't properly model long-term phase or amplitude drifts, which
might be systematic and correlated.

The second method for going beyond periodic is to replace the periodic
model with a nearly-periodic Gaussian process (cite Rasmussen).  This
is in fact precisely the right model if the aperiodicity of the
oscillator can be modeled by a classic mechanical damping and random
driving (cite Hou).  This method is powerful and simple but it is not
trivial, because the machinery of Gaussian processes requires some
software infrastructure.  We won't consider it further here.

\end{document}

